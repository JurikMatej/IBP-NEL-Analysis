;\chapter{Network Error Logging a relevantné technológie}
\label{nel-and-related-technologies}

TODO:
\begin{enumerate}
    \item Spísať štruktúru
    \item Prezrieť starý obsah pre inšpiráciu
    \item Prezrieť opravu po Petrovi a po vedúcom pre uľahčenie písania
    \item Postupne spísať po Report API
    \item Naštudovať podklady k Report API LEPŠIE, aby som presne vedel čo chcem spomenúť
\end{enumerate}


....


\section{DNS a zahrnutá Terminológia}
\begin{itemize}
    \item Doména
    \item Sub-doména
    \item Pay-level doména (vymazať odvšadiaľ a nepoužívať)
    \item Public Suffix
    \item eTLD a ich zoznam
    \item URLs
    https://url.spec.whatwg.org/
\end{itemize}

\section{Protokol HTTP}
\begin{itemize}
    \item Úvod 
    Klient Server komunikácia
    \item HTTP spec
    Network Partition Key, Request, Response, Status, Endpoint...
    \\
    Specs - https://httpwg.org/specs/rfc7230.html, https://httpwg.org/specs/rfc7231.html
    \item Fetch spec
    \item Media Type
    JSON spec = https://www.rfc-editor.org/rfc/rfc7159
    \item Obsah
    HTML - https://dom.spec.whatwg.org/, https://html.spec.whatwg.org/multipage/ 
    \\
    JS - https://tc39.es/ecma262/multipage/
    \item Security
    https://www.w3.org/TR/secure-contexts/
\end{itemize}

\section{Zlyhania v komunikácií Klient-Server}
\begin{itemize}
    \item General intro and info
    \item DNS
    \item Transport
    \item Application
    \item Special
    \item Web Reliability
    \item Existujúce riešenia
\end{itemize}

\section{Report API}
\begin{itemize}
    \item Všeobecná náväznosť na zlyhania komunikácie
    \item Cieľ
    \item Ako funguje
    \item Konfigurácia
    \item Príklady
    \item Možnosti reálneho využitia (konkrétna náväznosť na NEL)
\end{itemize}

\section{Network Error Logging}
\begin{itemize}
    \item Cieľ
    \item Ako zapadá do a ako využíva predošle spomenuté koncepty
    \item Ako funguje
    \item Konfigurácia
    \item Príklady
\end{itemize}


\section{OLD CONTENT}
V tejto kapitole popisujem problematiku, v ktorej sa táto práca venuje. Jedná sa tu o možné nástrahy pri komunikácií typu klient-server,
ktoré môžu nastať a spôsobiť probmlémy ako napríklad nedosiahnuteľnosť servera, s ktorým klient komunikáciu pôvodne nadviazal.
O takýchto a podobných problémoch sa server ziaľ nemá ako dozvedieť, a preto vzniká potreba nájsť nejaký spôsob ako takéto problémy pri ich
vzniku identifikovať a nahlásiť formou štruktúrovanej správy vývojárom zodpovedným za prevádzku daného servera. V prípade, že 
na problém takto poukázané, v nasledujúcich krokoch je možné postaviť sa k nemu s vhodnými protiopatreniami a úspešne ho odstrániť, a teda 
tak zároveň aj znovuspojazdniť predtým nefunkčnú komunkikáciu s klientami, u ktorých sa tento problém prejavoval.
\\
V tejto kapitole začnem s úvodom do širšieh spektra zavedenej problematiky, kde spomeniem technológie s podobným účelom 
ako Network Error Logging (ďalej označovaný iba ako NEL), aké nedostatky aktuálneho stavu problematiky riešia, no hlavne, aké majú nedostatky. 
Tým sa prepracujem k podstate a motivácií k nasadeniu samotného NEL-u a ďalších technológií priamo súvisiacich s ním. 
Jedná sa tu čisto o teoretický základ nutný pre pochopenie nasledujúcich sekcií tejto práce. 
Aktuálny stav využívania technológií spomínaných nižšie bude uvedený a detailne rozpísaný v kapitole \ref{related-work}.

\section{Všeobecná problematika zlyhaní v komunikáciach typu klient-server}

Aké problémy môžu nastať
\\
Ako sa to prejavuje
\\
Dôkaz, že server takéto problémy nedokáže detekovať

\section{Monitoriovanie spoľahlivosti webových služieb}

Sekcia bude vyplývať z článku priloženého k odporučenej literatúre zadania tejto BP \cite{nel-client-side-measurement-e2e-reliability}
\\
Existujúce populárne spôsoby riešenia uvedených problémov - napr: DVP, JavaScript...
\\
Aké problémy tieto riešenia stále nepokrývajú - napr. JavaScript sa vôbec nemusí spustiť (klient ho neobdrží od servera).
\\
Motivácia používania NEL - aké aktuálne riešenia ponúka pre možné chyby a perspektíva do budúcnosti.

\todo{Table 1: Properties satisfied by different approaches for detecting service reachability problems at scale.}

Odkaz na Table 1 - Jednou z motivácií pre vývoj štandardu NEL bolo, že ako jediný by bol schopný presne vyčísliť koľko klientov je
v danom čase (aktuálne) afektovaných výpadkom v komunikácií. Práve táto schopnosť ručí za pridanú hodnotu korektného určenia závažnosti,
tým pádom aj priority konkrétneho zlyhania. Správci serveru, ktorý interne používa NEL, majú dostatočný prehľad o stave siete a na základe 
toho môžu rozhodovať o tom, kam zamerajú svoje úsilie o riešenie závad.    

\section{Network Error Logging}

\todo{Náhľad na NEL z hladiska jeho špecifikácie}

NEL je štandard pre zachytávanie a získavanie chýb a zlyhaní na úrovni webového prehliadača navrhnutý organizáciou World Wide Web 
Consortium (ďalej v skratke iba W3C). Tento štandard bol prvýkrát popísaný v článku publikovanom 11/02/2014 a je dodnes aktualizovávaný
s tým, že posledná verzia jeho špecifikácie bola vydaná práve v dobe vzniku tejto bakalárskej práce, a to presne 5/10/2023. 

Hlavným cieľom NEL je poskytovať prevádzkovateľom webových serverov priamy pohľad na chybové stavy, ktoré môžu vzniknúť pri snahe 
klientských aplikácií komunikovať s nimi. Pri takejto klient-server komunikácií môže nastať hneď niekoľko kategórií problémov, 
kde v každej z nich si zaslúžia jednotlivé chyby svoje vysvetlenie. Dôležité je, že server v obyčajnom scenári takejto komunikácie 
nemá žiadnu možnosť dostať sa k informácií o tom, že sa niečo pokazilo, ani čo konkrétne bol samotný problém. Práve toto je cieľom 
napraviť pre tento štandard a tým pádom sa tvorcom tejto technológie jedná o postupné zvýšenie dostupnosti služieb poskytovaných 
na internete sprevádzkovaním NEL na svojich webových serveroch.

V nasledujúcich sekciách chcem vysvetliť jednotlivé detaily týkajúce sa tejto technológie, ktoré sú naprosto potrebné k porozumeniu 
implementačnej časti tejto práce - kapitole č.\ref{analysis-and-its-results}. Pokiaľ nebude uvedené inak, v tejto kapitole čerpám hlavne 
z poslednej dostupnej verzie dokumentu špecifikácie NEL\cite{W3C-NEL}.

\subsection{Základný model NEL}
\todo{Špecifikácia NEL, ktorá je jednoznačne potrebná pre účel pochopenia implementačnej časti (analýzy)} 

NEL je možné využiť pri komunikácií klienta so serverom pomocou protokolu HTTP. Jeho funkcionalita je zapracovaná do užívateľských 
prehliadačov založených na open-source?? distribúcií projektu \textbf{Chromium}, a to menovite napríklad: Google Chrome, Opera alebo 
Microsoft Edge. \todo{CITE}

Jeho vnútorné mechanizmy sú spustené práve vtedy, keď server pri vyhotovovaní novej požiadavky zašle v svojej odpovedi spolu s ostatnými 
aj hlavičku \code{NEL}. Tento proces sa nazýva \textbf{Policy delivery} a sprostredkuje dohodu o zbieraní, udržiavaní a nahlasovaní chýb 
vzniknutých pri komunikácií. Samotné politiky sú popísané vo väčšom detaile v sekcií \ref{nel-policies}.
\\
\\
Hlavička \code{NEL} vo svojej najjednoduchšej podobe musí obsahovať nasledovné položky: 

\begin{itemize}
    \item \code{report\textunderscore to} - menom označená skupina zberačov reportov (collerctors, viď \ref{reporting-api}) 
    \item \code{max\textunderscore age} - doba platnosti zaslanej politiky NEL
\end{itemize}

Tieto hlavičky musia byť zadané vo formáte \code{application\textbackslash json}, takže ukážkový obsah HTTP hlavičky NEL môže vyzerať 
napríklad takto:

\begin{lstlisting}[caption={Ukážka obsahu najjednoduchšej/minimálnej HTTP hlavičky NEL. Akékoľvek chyby budú hlásené do skupiny 
    \code{network-errors} po dobu platnosti tejto politiky, ktorá bola nastavená na 7 dní (604 800 / 60s / 60min / 24h)}]

{"report_to": "network-errors", "max_age": 604800}

\end{lstlisting}
% \label{nel-simple-example}
%\todo{center only the lstlisting without the caption}
%\\
\todo{V návrhu NEL je v detaile opísané, ako požiadavky na bezpečnosť ovplyvnili jeho návrh a funkcionalitu. Toto je nutné spomenúť, no myslím,
že sa v tejto práci tomu nie je potreba venovať}

\subsection{Konfigurácia}

Všetky položky, ktoré je možné v NEL hlavičke uviezť.

\subsection{Politiky a ich uskladňovanie na strane klienta}
\label{nel-policies}


\subsection{Rozšírenia plánované v budúcnosti}

Nová verzia NEL, ktorá bude schopná spolupracovať s Reporting API v1 (momentálne funguje iba s v0)



\section{Reporting API}
\label{reporting-api}

Doplnkové info, ktoré je skrátka nutné spomenúť \cite{W3C-reporting-api}
