\chapter{Záver}
\label{zaver}

Cieľom tejto práce bolo navrhnúť spôsob analýzy využitia technológie zvanej Network Error Logging (NEL), pričom bolo nutné zohľadniť podobné existujúce práce aby táto analýza prinášala nové poznatky.
Pri vytváraní návrhu som sa zoznámil so samotnou skúmanou technológiou, ale aj s nástrojmi, ktoré je možné využiť na jej skúmanie.
Menovite som sa zameral na projekt HTTP Archive, ktorý uverejňuje na platforme Google Cloud dáta použiteľné na získanie informácií o využití NEL na webe.
Keďže som zistil, že je však tento projekt nedostatočným podkladom pre analýzu skúmanej technológie, zameral som sa taktiež na Selenium.
S využitím Selenium je možné prehľadávať web v reálnom čase a popri tom ukladať informácie naznačujúce využitie NEL na jednotlivých prehľadávaných webstránkach.

Taktiež som zistil, že predtým ako môžem analýzu zhotoviť, budem musieť získať rozsiahly zoznam domén web serverov, ktoré budem skúmať. 
S ohľadom na túto skutočnosť som prehľadal dostupné možnosti zaobstarávania si zoznamov domén, pri čom som našiel projekt TRANCO.

Na základe získaných vedomostí som vymyslel návrh pre vykonanie vlastnej analýzy.
Jej výsledkom bude popis dôležitých metrík týkajúcich sa využitia NEL, ako napríklad percentuálny pomer domén, na ktorých sa NEL úspešne detegoval, k doménam, na ktorých sa nenachádzal.
Ďalej uvádzam, že ako dodatočnú súčasť práce je vhodné vytvoriť balík vizualizačných funkcií pre tvorbu grafickej reprezentácie získaných metrík.
Taká výstupná forma mojej analýzy má totiž potenciál stať sa súčasťou každoročného hlásenia o stave webu zvanom Web Almanac. 
