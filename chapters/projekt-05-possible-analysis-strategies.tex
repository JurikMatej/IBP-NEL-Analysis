\chapter{Návrh analýzy}
\label{possible-analysis-strategies}

V tejto kapitole popisujem návrh na vypracovanie analýzy nasadenia a využívania technológie NEL.

\subsubsection{Cieľ a rozsah}

Cieľom práce je analyzovať nasadenie technológie NEL na webe. 
Vďaka dostupným zdrojom dát popisovaných v kapitole \ref{data-sources-available-for-research} je možné skúmať stav využívania NEL:
\begin{enumerate}
    \item v histórií prehliadania webu zaznamenanej v dátach poskytovaných projektom HTTP Archive,
    \item v súčasnosti, použitím automatizovaného webového prehliadania.
\end{enumerate}

Na základe konzultácie s vedúcim tejto práce, pánom Polčákom, som sa rozhodol analýzu zhotoviť pre čo 
najrozsiahlejšie obdobie.
Zámerom je pokúsiť sa pokryť celé obdobie, za ktoré sa NEL doteraz používalo, no závisí na zdrojoch použitých pre analýzu, či budú všetky potrebné dáta dostupné.
Ako počiatočný dátum môžem zvoliť 25. september 2018, kedy autori tejto technológie publikovali jej špecifikáciu (viď sekciu \ref{network-error-logging}).
Od tohto počiatočného dátumu chcem zhotoviť prieskum až po dátum obhajoby tejto práce -- máj 2024.

\subsubsection{Existujúca analýza}

Taktiež som sa v rámci konzultácií zameral na získanie cenných informácií a poučení z existujúcej analýzy z roku 2023,
ktorú zhotovil pán Polčák spolu s jeho kolegom, pánom Jeřábkom \cite{nel-http-archive}.
Je zameraná na získanie metrík ako celkové využívanie NEL na množine skúmaných domén (viď obrázok \ref{fig:polcak-analysis}), aké NEL kolektory dané domény používajú a v akých konfiguráciách sa nasadené NEL vyskytuje.
Časové obdobie, za ktoré bola analýza zhotovená, tvorili februárové mesiace každého roku od 2018 do 2023. 
Za toto obdobie sa im podarilo zistiť, že počiatočné využitie NEL bolo 0\% a do februára roku 2023 stúplo na 11.73\% skúmaných domén.
Toto percento predstavuje z celkového počtu skúmaných domén 2 247 233 jedinečných domén.

\begin{figure}[!htb]
\begin{center}
    % \includegraphics[width=0.5\linewidth]{obrazky-figures/polcak-analysis.png}
    \includegraphics[width=0.55\linewidth]{obrazky-figures/polcak-analysis-hd.pdf}
    \caption{Využitie NEL na skúmaných doménach v analýze vypracovanej vo vedeckom článku Ing. Libora Polčáka Ph.D. a Ing. Kamila Jeřábka \cite{nel-http-archive}.}
    \label{fig:polcak-analysis}
\end{center}
\end{figure}

\pagebreak

Zistil som, že ako zdroj dát bol použitý HTTP Archive, ktorým sa zaoberá aj moja práca.
Je však dôležité poznamenať, že predošlá analýza bola vypracovaná iba za pomoci využitia 
práve jedného zdroja dostupného na každej skúmanej doméne.
Keďže bol v rámci práce použitý projekt HTTP Archive ako zdroj dát, k tomuto rozhodnutiu obmedziť sa mohlo viesť to, 
že ním využívaná služba BigQuery je spoplatnená (viď sekciu \ref{httparchive-costs}). 
Je pravdepodobné, že autori nemali prostriedky pre financovanie sťahovania takého veľkého objemu dát, 
ktorý predstavuje skúmané obdobie šiestich rokov (od 2018 do 2023).
Vo svojej práci usúdili, že vzhľadom na túto limitáciu je nutné vypracovať podrobnejšiu analýzu, 
doplnenú o dáta získané alternatívnym spôsobom \cite{nel-http-archive}.
Práve z toho dôvodu som počiatočne usúdil, že je nutné v mojej analýze okrem použitia HTTP Archive navrhnúť 
aj iný spôsob skúmania stavu využitia technológie NEL.

\subsubsection{Prieskum možností pre vylepšenia}

Preskúmal som spôsob, ktorý vo vyššie uvedenej práci použili na získavanie HTTP Archive dát.
Bol použitý nástroj pre sťahovanie dát z Google Cloud BigQuery, ktorý naprogramoval pán 
Jeřábek\footnote{\href{https://github.com/kjerabek/nel-http-archive-analysis}{https://github.com/kjerabek/nel-http-archive-analysis}}.
Počas testovania tohto nástroja som zistil, že je možné vylepšiť ho tak, aby som odstránil predošlú, v skutočnosti iba myslenú limitáciu s nedostatkom dát kvôli financiám.
Postupným skúšaním používania a úprav príkazu GoogleSQL použitého v spomínanom nástroji na extrahovanie dát som prišiel na skutočnosť, že BigQuery
si nárokuje poplatky nie za sťahovanie, ale za čítanie dát z jednotlivých stĺpcov svojich tabuliek.
To pre mňa znamenalo, že nebudem platiť za objem dát, ktoré stiahnem, ale za objem dát, ktoré pri spustení príkazu GoogleSQL bude musieť služba prečítať.
Prišiel som teda na to, že rozhodnutie obmedziť sa na jeden zdroj pre každú skúmanú doménu v predošlej analýze nezaručilo nižšiu spotrebu financií.
Tým pádom sa pre mňa otvorila možnosť naplno využiť HTTP Archive dáta úpravou spomínaného príkazu GoogleSQL, pričom by som na to potreboval rovnakú sumu peňazí.
Táto úprava spočívala v odstránení filtrov z tohto príkazu, ktoré jeho výsledné dáta limitovali na jeden zdroj pre každú doménu.

Google Cloud ako platforma pre BigQuery v dobe vypracovania mojej práce ponúkala novo zaregistrovaným účtom zadarmo skúšobný
kredit 300\$. 
Podľa BigQuery cenníka viem, že 1 terabajt prečítaných dát stojí 6.25\$.
To by znamenalo, že s využitím ponúkaného kreditu mám dostupnú kapacitu 48 terabajtov, 
pričom s každým uplynulým mesiacom získam ďalší terabajt naviac (viď ceny služby BigQuery v sekcií \ref{httparchive-costs}).

\subsubsection{Použité metódy}

Vzhľadom na možnosť vylepšenia, ktoré by garantovalo plnú využiteľnosť HTTP Archive dát počas celého zvoleného časového obdobia a výšky zadarmo dostupného kreditu
som sa rozhodol založiť celú svoju analýzu práve na HTTP Archive.

Druhou metódou skúmania stavu nasadenia NEL, automatizovaným prehliadaním webu, je možné ďalej doplniť moju prácu.
Vďaka nej totiž môžem získať najaktuálnejší prehľad nasadenia NEL na skúmaných doménach medzi tým ako sú publikované výsledky prehliadania webu od HTTP Archive.
Avšak je zložité pripraviť výpočtovú silu a vhodnú stratégiu prehľadávania webu na priblíženie sa k rozsahu a rýchlosti dosahovanej alternatívnou metódou.
Výsledok vývoja nového nástroja pre automatizované prehliadanie webu sa teda za čas vyhradený pre túto prácu nemôže vyrovnať existujúcemu komplexnému riešeniu, ktorým je HTTP Archive.
Je však možné taký nástroj aj vo veľmi jednoduchom prevedení využiť na kontrolu výsledkov, ktoré získam z dát HTTP Archive.
Preto som sa rozhodol v rámci mojej analýzy taký nástroj zhotoviť.
Kontrolovať bude do akej miery sa výsledky z najaktuálnejších, alternatívne získaných dát zhodujú s jeho vlastnými výsledkami.

\subsubsection{Vstupy a výstupy}

Keďže sa stal primárnym zdrojom dát pre moju analýzu HTTP Archive, jednoznačným výberom množiny skúmaných domén sa stal ním používaný Chrome User Experience Report. 
Vďaka tomu, že tabuľky HTTP Archive už tieto domény obsahujú, nie je nutné vynakladať úsilie naviac pre ich zaobstarávanie v samostatnom kroku. 
Preto som sa rozhodol nevyužiť TRANCO ako hlavný zoznam skúmaných domén.
Pri celkovej analýze ale tento rebríček môže byť stále užitočný pre získanie pohľadu na využívanie a detaily používania NEL tými najpopulárnejšími doménami z celkovej skúmanej množiny.

V dôsledku takto sformovanej stratégie pre analýzu využitia technológie NEL sa jej vstupmi stávajú dáta získané z HTTP Archive a spomínaného nástroja na kontrolu najaktuálnejšieho stavu.
Tieto dáta som sa rozhodol z oboch zdrojov štruktúrovať rovnako, aby ich analýza mohla prebiehať tým istým spôsobom.
Na základe sformovanej štruktúry týchto dát implementujem nástroj pre ich spracovanie, ktorý z nich vypočíta niekoľko metrík popisujúcich rôzne aspekty využívania technológie NEL.
Na základe obsahu skúmaných hlavičiek \code{NEL} a \code{Report-To}, som sa rozhodol zamerať sa na výpočet nasledujúcich metrík:
\begin{itemize}
    \item počet domén, ktoré NEL používajú,
    \item NEL kolektory, ktoré sa používajú, ich počet a najpoužívanejšie z nich,
    \item konfigurácie, v ktorých sa vyskytuje používaný NEL,
    \item pomer monitorovaných zdrojov k všetkým dostupným zdrojom na danej doméne,
    \item typy monitorovaných zdrojov,
    \item nesprávne nasadenie NEL.
\end{itemize}

Výsledkom celkovej mojej analýzy bude prezentovanie výsledných metrík vo vhodnej forme s poznamenaním a vysvetlením zistených skutočností a zaujímavostí. 

\pagebreak

\begin{figure}[!htb]
\begin{center}
 % \includegraphics[scale=0.48]{obrazky-figures/analysis-activity-diagram.png}    
 % \includegraphics[scale=0.65]{obrazky-figures/analysis-activity-diagram-hd.pdf}    
 \includegraphics[scale=0.65]{obrazky-figures/analysis-activity-diagram-revised.pdf}    
 \caption{Diagram aktivít analýzy.}
 \label{img:analysis-activity-diagram}
\end{center}
\end{figure}
