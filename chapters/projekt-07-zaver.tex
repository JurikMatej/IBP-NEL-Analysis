\chapter{Záver}
\label{zaver}

Cieľom tejto práce bolo navrhnúť spôsob analýzy využitia technológie zvanej Network Error Logging (NEL), pričom bolo nutné zohľadniť predošlú existujúcu prácu, na ktorú mala táto práca nadviazať novými poznatkami.
Výsledkom je rozsiahlejšia a podrobnejšia analýza, \mbox{v rámci} ktorej bol preskúmaný stav od septembra 2018 do apríla 2024.

Pre vykonanie tejto analýzy bolo implementovaných niekoľko nástrojov automatizujúcich procesy získavania vstupných dát, ich spracovanie na navrhnuté metriky a ich prezentovanie.
Proces získavania dát bol implementovaný ako sťahovanie dát projektu HTTP Archive, pričom tieto dáta boli doplnené o dáta získané automatizovaným prehliadaním webu nástrojom na to zhotoveným v rámci tejto práce. 
Analýza navrhnutých metrík prebehla skúmaním počtov domén s nasadeným NEL, počtov poskytovateľov NEL kolektorov, a tiež vyskytujúcich sa konfigurácií NEL a monitorovaných zdrojov.
Výsledky analýzy boli prezentované ako novo nadobudnuté poznatky.

Vzhľadom na to, že sa v tejto práci podarilo získať všetky dáta, ktoré sú potrebné na podrobnú analýzu využívania technológie NEL za preskúmané obdobie, nie je nutné pokračovať v získavaní zdrojových dát.
Avšak, určite je možné preskúmať iné, potencionálne lepšie stratégie použiteľné pre implementovaný nástroj na automatizované prehliadanie webu, ktorý analýzu môže vykonávať priebežne.
Tiež je pravdepodobné, že v blízkej budúcnosti bude publikovaná nová špecifikácia NEL, ktorá bude fungovať spolu s novšou verziou Reporting API, a teda bude nutné tento nástroj upraviť.

Všetky získané dáta boli odovzdané vedúcemu tejto práce.
Spolu s nimi budú naďalej používané aj nástroje implementované v rámci tejto práce na pokračovanie vykonávania analýzy za účelom vypracovania vedeckého článku týkajúceho sa využívania technológie NEL. 
