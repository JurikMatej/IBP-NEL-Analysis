\chapter{Zdroje dát potrebných pre analýzu}
\label{data-sources-available-for-research}

Zdroje dát, ich umiestnenie, formát, veľkosť, pokrytie, vhodnosť.
\\
Spôsoby ich získavania (možné a zvolené) budú popísané v kapitole \ref{possible-analysis-strategies}


\section{GCP BQ a HTTP Archive (+ iné zdroje v prípade potreby)}

KJerabek Repository

\section{Teória k čomukoľvek ďalšiemu, čo sa bude používať}

Našiel som napríklad \textbf{crawler.ninja} - web crawled data od 09.05.2021 do 1.10.2023 dostupné na stiahnutie (ale zdroj sú asi stále HTTP Archive dáta)

\section{Nástroje pre automatickú analýzu stavu nasadenia NEL - teória (selenium, bs4, pandas...)}

Selenium teória apod.
