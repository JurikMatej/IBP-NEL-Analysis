\chapter{Úvod}
\label{uvod}
\vspace{1cm}

V modernom svete webových technológií čoraz viac narastá počet možností predstavujúcich systémy a rámce, ktoré 
poskytujú ich užívateľovi, alebo vývojárovi spôsob tvorby nového produktu -- svojho webu. Tieto webové technológie sa 
od svojho vzniku prirodzene vyvinuli naozaj badateľne. Od jednoduchých stránok, za ktorými stál iba značkovací kód 
HTML sme sa postupne prepracovali cez kaskádové štýly CSS a vdýchnutie života prostredníctvom skriptovacieho jazyku 
JavaSript až k najnovším rámcovým technológiam súčasnosti. Po ceste sa k tomuto technologickému jadru pridali aj rôzne 
nové nápady týkajúce sa samotnej architektúry celkovej aplikácie, ktorá má vzniknúť. Práve teraz sú populárne 
napríklad takzvaná trojvrstvová architektúra (3-Tier Architecture) alebo mikroslužby (Microservices). No žiaľ aj s 
takýmto pokrokom vpred na strane vývoja webu sa stále stretávame s celkom bežnými problémami, ktoré môžu vzniknúť, aj 
keď použijeme tú najvhodnejšiu sadu technológií, ktorá je momentálne dostupná.

Jedná sa o problémy dostupnosti webovej stránky, ktorú chceme uverejniť na internete. Je totiž možné, že
potencionálny návštevník hocijakého webu narazí na problém ešte predtým ako na daný web vlastne reálne zavíta tým, že 
mu bude v prehliadači navrátená odpoveď s obsahom domovskej stránky. Medzi prekážky, na ktoré môžeme naraziť patrí 
napríklad už zlyhanie DNS servera pri rezolúcií IP adresy cieľovej destinácie. Alebo sa môže stať, že sa návštevník 
stretne so zádrhlom, ktorý spôsobí neplatný certifikát identity na našej stránke a jednoducho sa ďalej v takom prípade 
nedostane. Medzi podobné nedostatky patrí aj zaujímavý prípad, kedy je všetko týkajúce sa spoľahlivosti webstránky po 
technickej stránke v poriadku, no jednoducho trvá príliš dlho, dokedy sa domovská stránka načíta, a preto sa 
návštevník rozhodne odísť.

Riešenie takýchto a mnoho ďalších zádrhlov je tématikou diskuzie o spoľahlivosti jednotlivých webstránok (používa sa 
termín Website Reliability). V posledných rokoch sa takáto problematika dostáva do popredia a navrhujú sa rôzne 
spôsoby, akými sa možno vyvarovať spomínaným nástrahám. Jedným z takých spôsobov je nasadenie relatívne novej 
technológie Network Error Logging (NEL).

% \pagebreak

V tejto práci sa venujem špecificky technológií NEL a zameriavam sa na jej sprevádzkovanie na verejných stránkach 
internetu. V nasledujúcich troch kapitolách rozoberám do detailu, čo NEL predstavuje a ako funguje 
(kapitola \ref{nel-and-related-technologies}), kde a ako je možné dostať sa k dátam, ktoré svedčia o jeho nasadení a 
konfigurácií (kapitola \ref{data-sources-available-for-research}) a aké analytické práce už boli na túto a podobné témy
týkajúce sa NEL zhotovené a zverejnené (kapitola \ref{related-work}). Následne uvediem v kapitole 
\ref{possible-analysis-strategies} ako sa moja práca od predošlých odlišuje, ako je možné na ne nadviazať a čo všetko
v tejto mojej analýze sledujem a vyhodnocujem. 
% Zavŕšim zdokumentovaním procesu analýzy a jej konkrétnymi výsledkami a 
% implikáciami, ktoré z nej vyplývajú (\ref{analysis-and-its-results}). 
% Skripty a akékoľvek iné nástroje, ktoré boli pre zhotovenie tejto práce navrhnuté a 
% implementované budú popísané v kapitole \ref{tests-for-analysis-tools-created}, kde zhrniem ich použitie a podložím 
% ich správnosť príkladnými testami, ktorým boli podrobené. Práca zakončí zhrnutím nadobudnutých poznatkov v závere, ich 
% hodnotením a návrhmi na jej možné rozšírenia. 
