\chapter{Úvod}
\label{uvod}
\vspace{1cm}

V modernom svete webových technológií existuje veľký počet možností na tvorbu webových služieb. Čoraz viac narastá počet možností predstavujúcich systémy a rámce poskytujúce spôsob tvorby nového produktu v prostredí webu.
Tieto webové technológie sa od svojho vzniku, prirodzene, vyvinuli naozaj badateľne. 
Od jednoduchých stránok, za ktorými stál iba značkovací jazyk 
HTML sme sa postupne prepracovali cez kaskádové štýly skriptovací jazyk 
JavaSript až k súčasným technológiám ako React.js. 
Po ceste sa k tomuto technologickému jadru pridali aj rôzne 
nové nápady týkajúce sa architektúry celkovej webovej služby. 
Využíva sa napríklad takzvaná trojvrstvová architektúra (3-Tier Architecture) alebo mikroslužby (Microservices). 
No aj s takýmto pokrokom vpred na strane vývoja webu sa stále stretávame s celkom bežnými problémami, ktoré môžu vzniknúť, aj 
keď použijeme tú najvhodnejšiu sadu technológií, ktorá je momentálne dostupná.

Jedná sa napríklad o problémy dostupnosti zdrojov, ktoré chceme uverejniť na webe. 
Je totiž možné, že potencionálny záujemca o zdroj narazí na problém ešte predtým ako nadviaže spojenie s web serverom, na ktorom je zdroj uložený.
Medzi prekážky, na ktoré môžeme naraziť patrí napríklad už zlyhanie DNS servera pri preklade doménového mena na IP adresu cieľovej destinácie. 
Alebo sa môže stať, že sa návštevník stretne so zlyhaním, ktoré spôsobí neplatný certifikát identity nášho web serveru a spojenie s ním sa nevytvorí. 
Medzi podobné nedostatky patrí aj zaujímavý prípad, kedy je zdroj dostupný a web server zabezpečený v poriadku, no jednoducho trvá príliš dlho, kým sa domovská stránka načíta. Niekedy sa práve preto návštevník rozhodne odísť.

Riešenie takýchto a mnoho ďalších zlyhaní je témou na diskusiu o spoľahlivosti jednotlivých web serverov. 
V posledných rokoch sa takáto problematika dostáva do popredia a navrhujú sa rôzne 
spôsoby, akými sa možno vyvarovať spomínaným nástrahám. Jedným z takých spôsobov je nasadenie relatívne novej 
technológie Network Error Logging (NEL).

\pagebreak

V tejto práci sa venujem špecificky technológií NEL a zameriavam sa na analýzu jej nasadenia na širokom spektre web serverov. 
V nasledujúcich troch kapitolách detailne popisujem:
\begin{itemize}
    \item čo je to NEL a ako funguje (v kapitole \ref{nel-and-related-technologies}),
    \item kde a ako je možné dostať sa k dátam, ktoré obsahujú informácie o jeho nasadení a 
konfigurácií (v kapitole \ref{data-sources-available-for-research}),
    \item a aké analytické práce už boli na túto a podobné témy
týkajúce sa NEL zhotovené a zverejnené (v kapitole \ref{related-work}).
\end{itemize}

Na koniec uvádzam v kapitole 
\ref{possible-analysis-strategies} svoj vlastný návrh pre vyhodnotenie využívania tejto technológie a ako sa moja práca od predošlých odlišuje. 

% Analýzu zhodnotím a spolu s jej konkrétnymi výsledkami a 
% implikáciami, ktoré z nej vyplývajú (\ref{analysis-and-its-results}). 
% Skripty a akékoľvek iné nástroje, ktoré boli pre zhotovenie tejto práce navrhnuté a 
% implementované budú popísané v kapitole \ref{tests-for-analysis-tools-created}, kde zhrniem ich použitie a podložím 
% ich správnosť príkladnými testami, ktorým boli podrobené. Prácu zakončím zhrnutím nadobudnutých poznatkov v závere, ich 
% hodnotením a návrhmi na jej možné rozšírenia. 
