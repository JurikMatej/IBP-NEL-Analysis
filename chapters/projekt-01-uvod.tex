\chapter{Úvod}
\label{uvod}
\vspace{1cm}

V modernom svete webových technológií existuje veľký počet možností na tvorbu webových služieb. Čoraz viac narastá počet možností predstavujúcich systémy a rámce poskytujúce spôsob tvorby nového produktu v prostredí webu.
Tieto webové technológie sa od svojho vzniku, prirodzene, vyvinuli naozaj badateľne. 
Od jednoduchých stránok, za ktorými stál iba značkovací jazyk 
HTML sa web postupne prepracoval cez kaskádové štýly a skriptovací jazyk 
JavaSript až k súčasným technológiám ako React.js. 
Po ceste sa k tomuto technologickému jadru pridali aj rôzne 
nové nápady týkajúce sa architektúry celkovej webovej služby.
No aj s takýmto pokrokom vpred na strane vývoja webu sa stále stretávame s celkom bežnými problémami, ktoré môžu vzniknúť, aj 
keď použijeme tú najvhodnejšiu sadu technológií, ktorá je momentálne dostupná.

Jedná sa napríklad o problémy dostupnosti zdrojov, ktoré chceme uverejniť na webe. 
Je totiž možné, že potencionálny záujemca o zdroj narazí na problém ešte predtým ako nadviaže spojenie s web serverom, na ktorom je zdroj uložený.
Medzi prekážky, na ktoré môžeme naraziť patrí napríklad už zlyhanie DNS servera pri preklade doménového mena na IP adresu cieľovej destinácie. 
Alebo sa môže stať, že sa návštevník stretne so zlyhaním, ktoré spôsobí neplatný certifikát identity nášho web serveru a spojenie s ním sa nevytvorí. 
Medzi podobné nedostatky patrí aj prípad, kedy je zdroj dostupný a web server zabezpečený v poriadku, no jednoducho trvá príliš dlho, kým sa domovská stránka načíta. Niekedy sa práve preto návštevník rozhodne odísť.

Riešenie takýchto a mnoho ďalších zlyhaní je témou na diskusiu o spoľahlivosti jednotlivých web serverov. 
V posledných rokoch sa takáto problematika dostáva do popredia a navrhujú sa rôzne 
spôsoby, akými možno monitorovať spomínané problémy. Jedným z takých spôsobov je nasadenie relatívne novej 
technológie Network Error Logging (NEL).
V tejto práci sa venujem špecificky technológií NEL a zameriavam sa na analýzu jej nasadenia na širokom spektre webových serverov. 
V kapitole \ref{nel-and-related-technologies} predstavujem, čo je to NEL a ako funguje. V kapitole \ref{data-sources-available-for-research} zase uvádzam, kde a ako je možné dostať sa k dátam, ktoré obsahujú informácie o jeho nasadení.
Ďalej, kapitola \ref{possible-analysis-strategies} obsahuje môj návrh na vypracovanie analýzy nasadenia technológie NEL a kapitola \ref{analysis-and-its-results} po nej zahŕňa výsledky celkovej analýzy.
V rámci výsledkov predstavujem aj nástroje, ktoré boli pre ich získanie implementované. Záverom práce je zhrnutie nadobudnutých poznatkov, ich hodnotenie a návrhy na jej možné rozšírenia.
